% Options for packages loaded elsewhere
% Options for packages loaded elsewhere
\PassOptionsToPackage{unicode}{hyperref}
\PassOptionsToPackage{hyphens}{url}
\PassOptionsToPackage{dvipsnames,svgnames,x11names}{xcolor}
%
\documentclass[
  spanish,
  letterpaper,
  DIV=11,
  numbers=noendperiod,
  oneside]{scrartcl}
\usepackage{xcolor}
\usepackage[left=1in,marginparwidth=2.0666666666667in,textwidth=4.1333333333333in,marginparsep=0.3in]{geometry}
\usepackage{amsmath,amssymb}
\setcounter{secnumdepth}{-\maxdimen} % remove section numbering
\usepackage{iftex}
\ifPDFTeX
  \usepackage[T1]{fontenc}
  \usepackage[utf8]{inputenc}
  \usepackage{textcomp} % provide euro and other symbols
\else % if luatex or xetex
  \usepackage{unicode-math} % this also loads fontspec
  \defaultfontfeatures{Scale=MatchLowercase}
  \defaultfontfeatures[\rmfamily]{Ligatures=TeX,Scale=1}
\fi
\usepackage{lmodern}
\ifPDFTeX\else
  % xetex/luatex font selection
\fi
% Use upquote if available, for straight quotes in verbatim environments
\IfFileExists{upquote.sty}{\usepackage{upquote}}{}
\IfFileExists{microtype.sty}{% use microtype if available
  \usepackage[]{microtype}
  \UseMicrotypeSet[protrusion]{basicmath} % disable protrusion for tt fonts
}{}
\makeatletter
\@ifundefined{KOMAClassName}{% if non-KOMA class
  \IfFileExists{parskip.sty}{%
    \usepackage{parskip}
  }{% else
    \setlength{\parindent}{0pt}
    \setlength{\parskip}{6pt plus 2pt minus 1pt}}
}{% if KOMA class
  \KOMAoptions{parskip=half}}
\makeatother
% Make \paragraph and \subparagraph free-standing
\makeatletter
\ifx\paragraph\undefined\else
  \let\oldparagraph\paragraph
  \renewcommand{\paragraph}{
    \@ifstar
      \xxxParagraphStar
      \xxxParagraphNoStar
  }
  \newcommand{\xxxParagraphStar}[1]{\oldparagraph*{#1}\mbox{}}
  \newcommand{\xxxParagraphNoStar}[1]{\oldparagraph{#1}\mbox{}}
\fi
\ifx\subparagraph\undefined\else
  \let\oldsubparagraph\subparagraph
  \renewcommand{\subparagraph}{
    \@ifstar
      \xxxSubParagraphStar
      \xxxSubParagraphNoStar
  }
  \newcommand{\xxxSubParagraphStar}[1]{\oldsubparagraph*{#1}\mbox{}}
  \newcommand{\xxxSubParagraphNoStar}[1]{\oldsubparagraph{#1}\mbox{}}
\fi
\makeatother


\usepackage{longtable,booktabs,array}
\usepackage{calc} % for calculating minipage widths
% Correct order of tables after \paragraph or \subparagraph
\usepackage{etoolbox}
\makeatletter
\patchcmd\longtable{\par}{\if@noskipsec\mbox{}\fi\par}{}{}
\makeatother
% Allow footnotes in longtable head/foot
\IfFileExists{footnotehyper.sty}{\usepackage{footnotehyper}}{\usepackage{footnote}}
\makesavenoteenv{longtable}
\usepackage{graphicx}
\makeatletter
\newsavebox\pandoc@box
\newcommand*\pandocbounded[1]{% scales image to fit in text height/width
  \sbox\pandoc@box{#1}%
  \Gscale@div\@tempa{\textheight}{\dimexpr\ht\pandoc@box+\dp\pandoc@box\relax}%
  \Gscale@div\@tempb{\linewidth}{\wd\pandoc@box}%
  \ifdim\@tempb\p@<\@tempa\p@\let\@tempa\@tempb\fi% select the smaller of both
  \ifdim\@tempa\p@<\p@\scalebox{\@tempa}{\usebox\pandoc@box}%
  \else\usebox{\pandoc@box}%
  \fi%
}
% Set default figure placement to htbp
\def\fps@figure{htbp}
\makeatother



\ifLuaTeX
\usepackage[bidi=basic]{babel}
\else
\usepackage[bidi=default]{babel}
\fi
% get rid of language-specific shorthands (see #6817):
\let\LanguageShortHands\languageshorthands
\def\languageshorthands#1{}


\setlength{\emergencystretch}{3em} % prevent overfull lines

\providecommand{\tightlist}{%
  \setlength{\itemsep}{0pt}\setlength{\parskip}{0pt}}



 


\KOMAoption{captions}{tableheading}
\makeatletter
\@ifpackageloaded{caption}{}{\usepackage{caption}}
\AtBeginDocument{%
\ifdefined\contentsname
  \renewcommand*\contentsname{Tabla de contenidos}
\else
  \newcommand\contentsname{Tabla de contenidos}
\fi
\ifdefined\listfigurename
  \renewcommand*\listfigurename{Listado de Figuras}
\else
  \newcommand\listfigurename{Listado de Figuras}
\fi
\ifdefined\listtablename
  \renewcommand*\listtablename{Listado de Tablas}
\else
  \newcommand\listtablename{Listado de Tablas}
\fi
\ifdefined\figurename
  \renewcommand*\figurename{Figura}
\else
  \newcommand\figurename{Figura}
\fi
\ifdefined\tablename
  \renewcommand*\tablename{Tabla}
\else
  \newcommand\tablename{Tabla}
\fi
}
\@ifpackageloaded{float}{}{\usepackage{float}}
\floatstyle{ruled}
\@ifundefined{c@chapter}{\newfloat{codelisting}{h}{lop}}{\newfloat{codelisting}{h}{lop}[chapter]}
\floatname{codelisting}{Listado}
\newcommand*\listoflistings{\listof{codelisting}{Listado de Listados}}
\makeatother
\makeatletter
\makeatother
\makeatletter
\@ifpackageloaded{caption}{}{\usepackage{caption}}
\@ifpackageloaded{subcaption}{}{\usepackage{subcaption}}
\makeatother
\makeatletter
\@ifpackageloaded{sidenotes}{}{\usepackage{sidenotes}}
\@ifpackageloaded{marginnote}{}{\usepackage{marginnote}}
\makeatother
\usepackage{bookmark}
\IfFileExists{xurl.sty}{\usepackage{xurl}}{} % add URL line breaks if available
\urlstyle{same}
\hypersetup{
  pdfauthor={Equipo EDUMER},
  pdflang={es},
  colorlinks=true,
  linkcolor={blue},
  filecolor={Maroon},
  citecolor={Blue},
  urlcolor={Blue},
  pdfcreator={LaTeX via pandoc}}


\author{Equipo EDUMER}
\date{}
\begin{document}


\pandocbounded{\includegraphics[keepaspectratio]{images/coes.png}}

\pandocbounded{\includegraphics[keepaspectratio]{images/LOGO-OCS.png}}

\pandocbounded{\includegraphics[keepaspectratio]{images/qrcode.png}}

\section{\texorpdfstring{\textbf{Cohesión social y
migración}}{Cohesión social y migración}}\label{cohesiuxf3n-social-y-migraciuxf3n}

\subsection{\texorpdfstring{\textbf{Una década de cambios y
desafíos}}{Una década de cambios y desafíos}}\label{una-duxe9cada-de-cambios-y-desafuxedos}

\begin{center}\rule{0.5\linewidth}{0.5pt}\end{center}

\textbf{Roberto González\textsuperscript{1,2}}

\textbf{\textsuperscript{1}Centro de Estudios de Conflicto y Cohesión
Social - COES}

\textbf{\textsuperscript{2}Escuela de Psicología, Pontificia Universidad
Católica de Chile}

Segundo Foro de Cohesión Social COES - CEP

9 Septiembre 2025, Santiago

\section{Contexto y motivación}\label{contexto-y-motivaciuxf3n}

\subsection{Cohesión social}\label{cohesiuxf3n-social}

\subsection{Migración}\label{migraciuxf3n}

\subsection{ELSOC}\label{elsoc}

\section{Aceptación de la diversidad e
interculturalidad}\label{aceptaciuxf3n-de-la-diversidad-e-interculturalidad}

\subsection{Aceptación de la
diversidad}\label{aceptaciuxf3n-de-la-diversidad}

Es la actitud hacia personas de diferentes orígenes y culturas dentro de
una sociedad. Se mide evaluando los siguientes aspectos:

\begin{itemize}
\item
  \textbf{Simpatía hacia migrantes} --- ``¿Cuánto le agradan los
  {[}peruanos/haitianos/venezolanos{]} que viven en Chile?''\\
  \textbf{Opciones (5):} Muy poco o nada, Poco, Algo, Bastante, Mucho.\\
  \textbf{Recodificación (3):} Muy poco o nada; Poco o algo; Bastante o
  mucho.
\item
  \textbf{Amenaza laboral} --- ``Con la llegada de tantos {[}\ldots{]},
  en Chile está aumentando el desempleo.''\\
  \textbf{Opciones (5):} Totalmente en desacuerdo, En desacuerdo, Ni de
  acuerdo ni en desacuerdo, De acuerdo, Totalmente de acuerdo.\\
  \textbf{Recodificación (3):} En/Totalmente en desacuerdo; Ni de
  acuerdo ni en desacuerdo; De/Totalmente de acuerdo.
\end{itemize}

\subsection{Aceptación de la
diversidad}\label{aceptaciuxf3n-de-la-diversidad-1}

Es la actitud hacia personas de diferentes orígenes y culturas dentro de
una sociedad. Se mide evaluando los siguientes aspectos:

\begin{itemize}
\item
  \textbf{Pérdida de identidad nacional} --- ``Con la llegada de tantos
  {[}\ldots{]}, Chile está perdiendo su identidad.''\\
  \textbf{Opciones (5):} Totalmente en desacuerdo, En desacuerdo, Ni de
  acuerdo ni en desacuerdo, De acuerdo, Totalmente de acuerdo.\\
  \textbf{Recodificación (3):} En/Totalmente en desacuerdo; Ni de
  acuerdo ni en desacuerdo; De/Totalmente de acuerdo.
\item
  \textbf{Calidad del contacto} --- ``En los últimos 12 meses, cuando
  interactuó con {[}\ldots{]}, ¿cuán amistosa fue la experiencia?''\\
  \textbf{Opciones (5):} Muy poco amistosa, Poco amistosa, Ni amistosa
  ni no amistosa, Bastante amistosa, Muy amistosa.\\
  \textbf{Recodificación (3):} Bastante/Muy amistosa; Ni amistosa ni no
  amistosa; Muy/Poco amistosa.
\end{itemize}

\subsection{Aceptación de la
diversidad}\label{aceptaciuxf3n-de-la-diversidad-2}

Es la actitud hacia personas de diferentes orígenes y culturas dentro de
una sociedad. Se mide evaluando los siguientes aspectos:

\begin{itemize}
\tightlist
\item
  \textbf{Políticas más restrictivas} --- ``Chile debería tomar medidas
  más drásticas para impedir el ingreso de inmigrantes al país.''\\
  \textbf{Opciones (5):} Totalmente en desacuerdo, En desacuerdo, Ni de
  acuerdo ni en desacuerdo, De acuerdo, Totalmente de acuerdo.\\
  \textbf{Recodificación (3):} En/Totalmente en desacuerdo; Ni de
  acuerdo ni en desacuerdo; De/Totalmente de acuerdo.
\end{itemize}

\begin{center}\rule{0.5\linewidth}{0.5pt}\end{center}

\begin{center}\rule{0.5\linewidth}{0.5pt}\end{center}

\begin{center}\rule{0.5\linewidth}{0.5pt}\end{center}

\begin{center}\rule{0.5\linewidth}{0.5pt}\end{center}

\begin{center}\rule{0.5\linewidth}{0.5pt}\end{center}

aqui grafo iden nacional

\begin{center}\rule{0.5\linewidth}{0.5pt}\end{center}

\begin{center}\rule{0.5\linewidth}{0.5pt}\end{center}

\begin{center}\rule{0.5\linewidth}{0.5pt}\end{center}

\begin{center}\rule{0.5\linewidth}{0.5pt}\end{center}

\begin{center}\rule{0.5\linewidth}{0.5pt}\end{center}

\section{Cohesión social y
migración}\label{cohesiuxf3n-social-y-migraciuxf3n-1}

\section{Seguridad y migración}\label{seguridad-y-migraciuxf3n}

\section{Vinculos territoriales y
migración}\label{vinculos-territoriales-y-migraciuxf3n}

\section{Redes y migración}\label{redes-y-migraciuxf3n}

\subsection{Conclusiones}\label{conclusiones}

\textbf{1. Agenda de medición sobre meritocracia}: primera evidencia
respecto a una escala que permite evaluar la evolución de las creencias
meritocraticas en estudiantes en el tiempo

\textbf{2. Cohortes y etapas escolares}: no hay evidencia de invarianza
entre cohortes → \emph{diferencias en las maneras de comprender la
meritocracia según etapa escolar}

\textbf{3. Implicancias y proyecciones}: (i) profundizar en el rol del
talento, padres ricos y contactos, y (ii) analizar determinantes y
consecuencias de la escala de medición

\section{¡Gracias por su atención!}\label{gracias-por-su-atenciuxf3n}

\subsection{Referencias}\label{referencias}

\phantomsection\label{refs}




\end{document}
