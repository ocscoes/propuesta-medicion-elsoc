% Options for packages loaded elsewhere
% Options for packages loaded elsewhere
\PassOptionsToPackage{unicode}{hyperref}
\PassOptionsToPackage{hyphens}{url}
\PassOptionsToPackage{dvipsnames,svgnames,x11names}{xcolor}
%
\documentclass[
  spanish,
  letterpaper,
  DIV=11,
  numbers=noendperiod,
  oneside]{scrartcl}
\usepackage{xcolor}
\usepackage[left=1in,marginparwidth=2.0666666666667in,textwidth=4.1333333333333in,marginparsep=0.3in]{geometry}
\usepackage{amsmath,amssymb}
\setcounter{secnumdepth}{-\maxdimen} % remove section numbering
\usepackage{iftex}
\ifPDFTeX
  \usepackage[T1]{fontenc}
  \usepackage[utf8]{inputenc}
  \usepackage{textcomp} % provide euro and other symbols
\else % if luatex or xetex
  \usepackage{unicode-math} % this also loads fontspec
  \defaultfontfeatures{Scale=MatchLowercase}
  \defaultfontfeatures[\rmfamily]{Ligatures=TeX,Scale=1}
\fi
\usepackage{lmodern}
\ifPDFTeX\else
  % xetex/luatex font selection
\fi
% Use upquote if available, for straight quotes in verbatim environments
\IfFileExists{upquote.sty}{\usepackage{upquote}}{}
\IfFileExists{microtype.sty}{% use microtype if available
  \usepackage[]{microtype}
  \UseMicrotypeSet[protrusion]{basicmath} % disable protrusion for tt fonts
}{}
\makeatletter
\@ifundefined{KOMAClassName}{% if non-KOMA class
  \IfFileExists{parskip.sty}{%
    \usepackage{parskip}
  }{% else
    \setlength{\parindent}{0pt}
    \setlength{\parskip}{6pt plus 2pt minus 1pt}}
}{% if KOMA class
  \KOMAoptions{parskip=half}}
\makeatother
% Make \paragraph and \subparagraph free-standing
\makeatletter
\ifx\paragraph\undefined\else
  \let\oldparagraph\paragraph
  \renewcommand{\paragraph}{
    \@ifstar
      \xxxParagraphStar
      \xxxParagraphNoStar
  }
  \newcommand{\xxxParagraphStar}[1]{\oldparagraph*{#1}\mbox{}}
  \newcommand{\xxxParagraphNoStar}[1]{\oldparagraph{#1}\mbox{}}
\fi
\ifx\subparagraph\undefined\else
  \let\oldsubparagraph\subparagraph
  \renewcommand{\subparagraph}{
    \@ifstar
      \xxxSubParagraphStar
      \xxxSubParagraphNoStar
  }
  \newcommand{\xxxSubParagraphStar}[1]{\oldsubparagraph*{#1}\mbox{}}
  \newcommand{\xxxSubParagraphNoStar}[1]{\oldsubparagraph{#1}\mbox{}}
\fi
\makeatother


\usepackage{longtable,booktabs,array}
\usepackage{calc} % for calculating minipage widths
% Correct order of tables after \paragraph or \subparagraph
\usepackage{etoolbox}
\makeatletter
\patchcmd\longtable{\par}{\if@noskipsec\mbox{}\fi\par}{}{}
\makeatother
% Allow footnotes in longtable head/foot
\IfFileExists{footnotehyper.sty}{\usepackage{footnotehyper}}{\usepackage{footnote}}
\makesavenoteenv{longtable}
\usepackage{graphicx}
\makeatletter
\newsavebox\pandoc@box
\newcommand*\pandocbounded[1]{% scales image to fit in text height/width
  \sbox\pandoc@box{#1}%
  \Gscale@div\@tempa{\textheight}{\dimexpr\ht\pandoc@box+\dp\pandoc@box\relax}%
  \Gscale@div\@tempb{\linewidth}{\wd\pandoc@box}%
  \ifdim\@tempb\p@<\@tempa\p@\let\@tempa\@tempb\fi% select the smaller of both
  \ifdim\@tempa\p@<\p@\scalebox{\@tempa}{\usebox\pandoc@box}%
  \else\usebox{\pandoc@box}%
  \fi%
}
% Set default figure placement to htbp
\def\fps@figure{htbp}
\makeatother


% definitions for citeproc citations
\NewDocumentCommand\citeproctext{}{}
\NewDocumentCommand\citeproc{mm}{%
  \begingroup\def\citeproctext{#2}\cite{#1}\endgroup}
\makeatletter
 % allow citations to break across lines
 \let\@cite@ofmt\@firstofone
 % avoid brackets around text for \cite:
 \def\@biblabel#1{}
 \def\@cite#1#2{{#1\if@tempswa , #2\fi}}
\makeatother
\newlength{\cslhangindent}
\setlength{\cslhangindent}{1.5em}
\newlength{\csllabelwidth}
\setlength{\csllabelwidth}{3em}
\newenvironment{CSLReferences}[2] % #1 hanging-indent, #2 entry-spacing
 {\begin{list}{}{%
  \setlength{\itemindent}{0pt}
  \setlength{\leftmargin}{0pt}
  \setlength{\parsep}{0pt}
  % turn on hanging indent if param 1 is 1
  \ifodd #1
   \setlength{\leftmargin}{\cslhangindent}
   \setlength{\itemindent}{-1\cslhangindent}
  \fi
  % set entry spacing
  \setlength{\itemsep}{#2\baselineskip}}}
 {\end{list}}
\usepackage{calc}
\newcommand{\CSLBlock}[1]{\hfill\break\parbox[t]{\linewidth}{\strut\ignorespaces#1\strut}}
\newcommand{\CSLLeftMargin}[1]{\parbox[t]{\csllabelwidth}{\strut#1\strut}}
\newcommand{\CSLRightInline}[1]{\parbox[t]{\linewidth - \csllabelwidth}{\strut#1\strut}}
\newcommand{\CSLIndent}[1]{\hspace{\cslhangindent}#1}

\ifLuaTeX
\usepackage[bidi=basic]{babel}
\else
\usepackage[bidi=default]{babel}
\fi
% get rid of language-specific shorthands (see #6817):
\let\LanguageShortHands\languageshorthands
\def\languageshorthands#1{}


\setlength{\emergencystretch}{3em} % prevent overfull lines

\providecommand{\tightlist}{%
  \setlength{\itemsep}{0pt}\setlength{\parskip}{0pt}}



 


\KOMAoption{captions}{tableheading}
\makeatletter
\@ifpackageloaded{caption}{}{\usepackage{caption}}
\AtBeginDocument{%
\ifdefined\contentsname
  \renewcommand*\contentsname{Tabla de contenidos}
\else
  \newcommand\contentsname{Tabla de contenidos}
\fi
\ifdefined\listfigurename
  \renewcommand*\listfigurename{Listado de Figuras}
\else
  \newcommand\listfigurename{Listado de Figuras}
\fi
\ifdefined\listtablename
  \renewcommand*\listtablename{Listado de Tablas}
\else
  \newcommand\listtablename{Listado de Tablas}
\fi
\ifdefined\figurename
  \renewcommand*\figurename{Figura}
\else
  \newcommand\figurename{Figura}
\fi
\ifdefined\tablename
  \renewcommand*\tablename{Tabla}
\else
  \newcommand\tablename{Tabla}
\fi
}
\@ifpackageloaded{float}{}{\usepackage{float}}
\floatstyle{ruled}
\@ifundefined{c@chapter}{\newfloat{codelisting}{h}{lop}}{\newfloat{codelisting}{h}{lop}[chapter]}
\floatname{codelisting}{Listado}
\newcommand*\listoflistings{\listof{codelisting}{Listado de Listados}}
\makeatother
\makeatletter
\makeatother
\makeatletter
\@ifpackageloaded{caption}{}{\usepackage{caption}}
\@ifpackageloaded{subcaption}{}{\usepackage{subcaption}}
\makeatother
\makeatletter
\@ifpackageloaded{sidenotes}{}{\usepackage{sidenotes}}
\@ifpackageloaded{marginnote}{}{\usepackage{marginnote}}
\makeatother
\usepackage{bookmark}
\IfFileExists{xurl.sty}{\usepackage{xurl}}{} % add URL line breaks if available
\urlstyle{same}
\hypersetup{
  pdfauthor={Equipo EDUMER},
  pdflang={es},
  colorlinks=true,
  linkcolor={blue},
  filecolor={Maroon},
  citecolor={Blue},
  urlcolor={Blue},
  pdfcreator={LaTeX via pandoc}}


\author{Equipo EDUMER}
\date{}
\begin{document}


\pandocbounded{\includegraphics[keepaspectratio]{images/coes.png}}

\pandocbounded{\includegraphics[keepaspectratio]{images/LOGO-OCS.png}}

\pandocbounded{\includegraphics[keepaspectratio]{images/qrcode.png}}

\section{\texorpdfstring{\textbf{Cohesión social y
migración}:}{Cohesión social y migración:}}\label{cohesiuxf3n-social-y-migraciuxf3n}

\subsection{\texorpdfstring{\textbf{Una década de cambios y
desafíos}}{Una década de cambios y desafíos}}\label{una-duxe9cada-de-cambios-y-desafuxedos}

\begin{center}\rule{0.5\linewidth}{0.5pt}\end{center}

\textbf{Roberto González\textsuperscript{1,2}}

\textbf{\textsuperscript{1}Centro de Estudios de Conflicto y Cohesión
Social - COES}

\textbf{\textsuperscript{2}Escuela de Psicología, Pontificia Universidad
Católica de Chile}

Segundo Foro de Cohesión Social - COES/CEP

Santiago, 9 Septiembre 2025

\section{Contenidos}\label{contenidos}

\textbf{1. Contexto y motivación}

\textbf{2. Radiografía de actitudes hacia la migración}

\textbf{3. Cohesión social y migración}

\textbf{4. Conclusiones}

\section{1. Contexto y motivación}\label{contexto-y-motivaciuxf3n}

\subsection{¿Qué es la cohesión
social?}\label{quuxe9-es-la-cohesiuxf3n-social}

\begin{itemize}
\item
  Se relaciona con la cantidad y calidad de los vínculos existentes en
  una sociedad.
\item
  Múltiples dimensiones, que se pueden agrupar en dos grandes grupos
  (Chan et~al., 2006):

  \begin{itemize}
  \item
    Cohesión \textbf{vertical}: vínculos con instituciones y el estado
    (confianza en instituciones, participación política.
  \item
    Cohesión \textbf{horizontal}: vínculos con grupos, confianza
    interpersonal, redes de apoyo y seguridad.
  \end{itemize}
\item
  Esta presentación se centra en la relación entre la dimensión
  horizontal de la cohesión social y la migración.
\end{itemize}

\subsection{Migración en Chile}\label{migraciuxf3n-en-chile}

\begin{itemize}
\item
  \textbf{Crecimiento acelerado y cambio en la composición migratoria}:
  De acuerdo con las cifras del último CENSO (INE, 2024) la población
  extranjera en Chile alcanzó \textbf{1,6 millones} de personas (8,8\%
  del total), duplicando la cifra del censo 2017.
\item
  \textbf{Percepciones ambivalentes y polarización}:

  \begin{itemize}
  \tightlist
  \item
    Migrantes son vistos como trabajadores, pero también asociados a
    delincuencia.
  \item
    En zonas de mayor exposición surgen percepciones más negativas, en
    línea con la tesis de hostilidad inicial ante diversidad (Paolini
    et~al., 2014).
  \end{itemize}
\item
  \textbf{Impacto social y tensiones en la cohesión}: El aumento abrupto
  de la migración ha desafiado los marcos institucionales y generado
  tensiones en la cohesión social (Dinesen et~al., 2020; Reibold et~al.,
  2025),en particular a nivel de los \textbf{vínculos sociales}
  (cohesión horizontal).
\end{itemize}

\subsection{Investigando la cohesión
social}\label{investigando-la-cohesiuxf3n-social}

\hfill
\includegraphics[width=1\linewidth,height=\textheight,keepaspectratio]{images/LOGO-OCS.png}

\begin{itemize}
\item
  \href{https://ocs-coes.com/}{El Observatorio de Cohesión Social (OCS)}
  es un proyecto enmarcado en el FONDAP° 1523A0005:
  \href{https://coes.cl/}{Centro de Estudios de Conflicto y Cohesión
  Social}
\item
  Surge el año 2020 con el objetivo de contribuir al análisis de la
  cohesión social en Chile y América latina
\item
  Se basa en la experiencia de proyectos internacionales de
  conceptualización y medición de cohesión social
\end{itemize}

\subsection{}\label{section}

\hfill
\includegraphics[width=1\linewidth,height=\textheight,keepaspectratio]{images/elsoc.png}

\begin{itemize}
\item
  El \href{https://coes.cl/elsoc/}{Estudio Longitudinal Social de Chile}
  es un panel representativo desde el 2016 que busca analizar cómo
  piensan, sienten y se comportan las personas con respecto al conflicto
  y la cohesión social en Chile
\item
  Tiene un diseño muestral complejo: probabilístico, por conglomerados,
  multietápico y estratificado según el tamaño de las ciudades
\item
  Contiene baterías que tematizan el vínculo en sociedad que permiten
  medir y analizar las dimensiones de la cohesión social
\end{itemize}

\subsection{Medición de cohesión social horizontal con
ELSOC}\label{mediciuxf3n-de-cohesiuxf3n-social-horizontal-con-elsoc}

\pandocbounded{\includegraphics[keepaspectratio]{images/propuesta-nueva.png}}

La propuesta de medición de cohesión social horizontal con ELSOC se basa
en tres subdimensiones: seguridad, vínculos territoriales y redes
sociales.

\begin{itemize}
\item
  Seguridad se mide desde su plano subjetivo, a través de la percepción
  y satisfacción de seguridad con el barrio, y desde su plano objetivo,
  mediante la frecuencia de hechos violentos como asaltos, peleas
  callejeras y tráficos de drogas
\item
  Vínculos territoriales se mide por dos factores; la satisfacción con
  el barrio, tomando en cuenta elementos como la sociabilidad, facilidad
  de hacer amigos, y la cordialidad y colaboración entre vecinos; el
  sentido de pertenencia, el cual aborda elementos identitarios en
  relación al barrio donde vive en encuestado, tales como si es su
  barrio ideal, se identifica con él y si se siente integrado.
\item
  En redes sociales, comportamientos prosociales se mide por la
  asistencia a reuniones públicas y la realización de voluntariado.
  Ayuda económica considera el haber prestado dinero en algún momento a
  alguien y haber ayudado a encontrar trabajo a un tercero. Por último,
  confianza interpersonal considera confianza social generalizada, donde
  se pregunta si se puede o no confiar en la mayoría de las personas, y
  además altruismo social generalizado, el que pregunta si las personas
  generalmente tratan de ayudar a otros o se preocupan de sí mismas
\end{itemize}

\section{2. Aceptación de la diversidad e
interculturalidad}\label{aceptaciuxf3n-de-la-diversidad-e-interculturalidad}

\subsection{Aceptación de la
diversidad}\label{aceptaciuxf3n-de-la-diversidad}

Es la actitud hacia personas de diferentes orígenes y culturas dentro de
una sociedad. Se mide evaluando los siguientes aspectos:

\begin{itemize}
\item
  \textbf{Simpatía hacia migrantes} --- ``¿Cuánto le agradan los
  {[}peruanos/haitianos/venezolanos{]} que viven en Chile?''
\item
  \textbf{Amenaza laboral} --- ``Con la llegada de tantos {[}\ldots{]},
  en Chile está aumentando el desempleo.''
\item
  \textbf{Pérdida de identidad nacional} --- ``Con la llegada de tantos
  {[}\ldots{]}, Chile está perdiendo su identidad.''
\item
  \textbf{Calidad del contacto} --- ``En los últimos 12 meses, cuando
  interactuó con {[}\ldots{]}, ¿cuán amistosa fue la experiencia?''
\item
  \textbf{Políticas más restrictivas} --- ``Chile debería tomar medidas
  más drásticas para impedir el ingreso de inmigrantes al país.''
\end{itemize}

\begin{center}\rule{0.5\linewidth}{0.5pt}\end{center}

\begin{center}\rule{0.5\linewidth}{0.5pt}\end{center}

\begin{center}\rule{0.5\linewidth}{0.5pt}\end{center}

\begin{center}\rule{0.5\linewidth}{0.5pt}\end{center}

\begin{center}\rule{0.5\linewidth}{0.5pt}\end{center}

\begin{center}
\includegraphics[width=1\linewidth,height=\textheight,keepaspectratio]{cep_2025_files/figure-pdf/unnamed-chunk-2-1.pdf}
\end{center}

\begin{center}\rule{0.5\linewidth}{0.5pt}\end{center}

\begin{center}\rule{0.5\linewidth}{0.5pt}\end{center}

\begin{center}\rule{0.5\linewidth}{0.5pt}\end{center}

\begin{center}\rule{0.5\linewidth}{0.5pt}\end{center}

\begin{center}\rule{0.5\linewidth}{0.5pt}\end{center}

\section{3. Cohesión social horizontal y
migración}\label{cohesiuxf3n-social-horizontal-y-migraciuxf3n}

{\textbf{3.1. Seguridad}}

{\textbf{3.2. Vínculos territoriales}}

{\textbf{3.3. Redes}}

\section{3. Cohesión social horizontal y
migración}\label{cohesiuxf3n-social-horizontal-y-migraciuxf3n-1}

{\textbf{3.1. Seguridad}}

{\textbf{3.2. Vínculos territoriales}}

{\textbf{3.3. Redes}}

\begin{center}\rule{0.5\linewidth}{0.5pt}\end{center}

\begin{center}
\includegraphics[width=1\linewidth,height=\textheight,keepaspectratio]{cep_2025_files/figure-pdf/unnamed-chunk-4-1.pdf}
\end{center}

Al cruzar el acuerdo con políticas restrictivas hacia la migración y la
percepción de seguridad, se observa que la mayoría de quienes respaldan
estas medidas reporta niveles medianos o bajos de seguridad pública,
mientras solo una minoría percibe alta seguridad. Esta tendencia se
mantiene en todas las olas, con un aumento marcado de la inseguridad
percibida en 2023.

\begin{center}\rule{0.5\linewidth}{0.5pt}\end{center}

\begin{center}
\includegraphics[width=1\linewidth,height=\textheight,keepaspectratio]{cep_2025_files/figure-pdf/unnamed-chunk-5-1.pdf}
\end{center}

Al cruzar la simpatía hacia los migrantes con la percepción de
seguridad, se observa que la mayoría de quienes expresan muy poca o
ninguna simpatía reporta niveles medianos o bajos de seguridad pública,
mientras solo una minoría declara alta seguridad. Esta pauta se repite
en todas las olas y se intensifica en 2023, con un aumento significativo
de la inseguridad percibida.

\begin{center}\rule{0.5\linewidth}{0.5pt}\end{center}

\begin{center}
\includegraphics[width=1\linewidth,height=\textheight,keepaspectratio]{cep_2025_files/figure-pdf/unnamed-chunk-6-1.pdf}
\end{center}

Quienes han tenido un contacto poco amistoso con migrantes reportan
niveles medianos o bajos de seguridad pública, mientras solo una minoría
declara alta seguridad. En este grupo, la percepción de baja seguridad
ha variado en el tiempo. Ahora bien, en este grupo, la alta seguridad ha
ido a la baja consistentemente.

\section{3. Cohesión social horizontal y
migración}\label{cohesiuxf3n-social-horizontal-y-migraciuxf3n-2}

{\textbf{3.1. Seguridad}}

{\textbf{3.2. Vínculos territoriales}}

{\textbf{3.3. Redes}}

\begin{center}\rule{0.5\linewidth}{0.5pt}\end{center}

\begin{center}
\includegraphics[width=1\linewidth,height=\textheight,keepaspectratio]{cep_2025_files/figure-pdf/unnamed-chunk-7-1.pdf}
\end{center}

Dentro de quienes están muy de acuerdo con que la llegada de migrantes
ha generado una pérdida de identidad nacional, la mayoría reporta un
mediano sentido de pertenencia barrial, seguido por una baja pertenencia
y en menor mendida una alta pertenencia.

\begin{center}\rule{0.5\linewidth}{0.5pt}\end{center}

\begin{center}
\includegraphics[width=1\linewidth,height=\textheight,keepaspectratio]{cep_2025_files/figure-pdf/unnamed-chunk-8-1.pdf}
\end{center}

Dentro de quienes han tenido un contacto poco amistoso con migrantes, la
mayoría reporta una mediana y baja satisfacción barrial. Solo un grupo
minoritario reporta alta satisfacción con el barrio.

\begin{center}\rule{0.5\linewidth}{0.5pt}\end{center}

\begin{center}
\includegraphics[width=1\linewidth,height=\textheight,keepaspectratio]{cep_2025_files/figure-pdf/unnamed-chunk-9-1.pdf}
\end{center}

Dentro de quienes sienten poca simpatía hacia migrantes, la mayoría
reporta una mediana satisfacción barrial, seguido por una baja
satisfacción y en menor mendida una alta satisfacción.

\section{3. Cohesión social horizontal y
migración}\label{cohesiuxf3n-social-horizontal-y-migraciuxf3n-3}

{\textbf{3.1. Seguridad}}

{\textbf{3.2. Vínculos territoriales}}

{\textbf{3.3. Redes}}

\begin{center}\rule{0.5\linewidth}{0.5pt}\end{center}

\begin{center}
\includegraphics[width=1\linewidth,height=\textheight,keepaspectratio]{cep_2025_files/figure-pdf/unnamed-chunk-10-1.pdf}
\end{center}

Dentro de quienes están muy de acuerdo con políticas restrictivas para
el ingreso de migrantes, la mayoría presenta un bajo comportamiento
prosocial. El grupo con un alto comportamiento prosocial es minoritario
y aquel de mediano comportamiento va a la baja en el tiempo.

\begin{center}\rule{0.5\linewidth}{0.5pt}\end{center}

\begin{center}
\includegraphics[width=1\linewidth,height=\textheight,keepaspectratio]{cep_2025_files/figure-pdf/unnamed-chunk-11-1.pdf}
\end{center}

Quienes han tenido un contacto poco amistoso con migrantes registran, en
su mayoría, un bajo comportamiento prosocial. El mediano comportamiento
prosocial dentro de este grupo ha ido a la baja. En suma, hay una
relación negativa.

\begin{center}\rule{0.5\linewidth}{0.5pt}\end{center}

\begin{center}
\includegraphics[width=1\linewidth,height=\textheight,keepaspectratio]{cep_2025_files/figure-pdf/unnamed-chunk-12-1.pdf}
\end{center}

Quienes están muy de acuerdo con la pérdida de identidad nacional por la
llegada de migrantes registran, en su mayoría, un bajo comportamiento
prosocial. El mediano comportamiento prosocial dentro de este grupo ha
ido a la baja. En definitiva, hay una relación negativa.

\section{4. Conclusiones}\label{conclusiones}

\subsection{Conclusiones}\label{conclusiones-1}

\textbf{1. }:

\textbf{2. }:

\textbf{3. }:

\section{¡Gracias por su atención!}\label{gracias-por-su-atenciuxf3n}

\subsection{Referencias}\label{referencias}

\phantomsection\label{refs}
\begin{CSLReferences}{1}{0}
\bibitem[\citeproctext]{ref-chan_reconsidering_2006}
Chan, J., To, H.-P., \& Chan, E. (2006). Reconsidering {Social
Cohesion}: {Developing} a {Definition} and {Analytical Framework} for
{Empirical Research}. \emph{Social Indicators Research}, \emph{75}(2),
273-302. \url{https://doi.org/10.1007/s11205-005-2118-1}

\bibitem[\citeproctext]{ref-dinesen_trust_2020}
Dinesen, P. T., Schaeffer, M., \& Sønderskov, K. M. (2020). Ethnic
{Diversity} and {Social Trust}: {A Narrative} and {Meta-Analytical
Review}. \emph{Annual Review of Political Science}, \emph{23}(1),
441-465. \url{https://doi.org/10.1146/annurev-polisci-052918-020708}

\bibitem[\citeproctext]{ref-censo_2024}
INE. (2024). Resultados {Fecundidad}, Migraci{ó}n Interna e
Internacional {CENSO} 2024.

\bibitem[\citeproctext]{ref-paolini_intergroup_2014}
Paolini, S., Harwood, J., Rubin, M., Husnu, S., Joyce, N., \& Hewstone,
M. (2014). Positive and Extensive Intergroup Contact in the Past Buffers
against the Disproportionate Impact of Negative Contact in the Present.
\emph{European Journal of Social Psychology}, \emph{44}(6), 548-562.
\url{https://doi.org/10.1002/ejsp.2029}

\bibitem[\citeproctext]{ref-reibold_trust_2025}
Reibold, K., Bachvarova, M., \& Lenard, P. T. (2025). Introduction:
Trust, Social Cohesion, and Integration. \emph{Critical Review of
International Social and Political Philosophy}, 1-18.
\url{https://doi.org/10.1080/13698230.2025.2528379}

\end{CSLReferences}




\end{document}
